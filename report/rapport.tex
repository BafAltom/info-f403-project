\documentclass[a4paper,10pt]{article}
\usepackage[utf8]{inputenc}
\usepackage{url}
\usepackage{hyperref}
\usepackage{listings}
\usepackage{color}
\usepackage{verbatim}
\definecolor{grey}{rgb}{0.9,0.9,0.9}
\usepackage{float}
\usepackage{graphicx}
\usepackage{fancyhdr}
%\pagestyle{fancy} % voir si laisse ce style ou pas ?
%\usepackage[top=2.5cm,bottom=2.5cm,right=2.5cm,left=2.5cm]{geometry}
\usepackage[right=4cm,left=4cm]{geometry}
\lstset{
language=C++,
basicstyle=\footnotesize\fontfamily{pcr},
backgroundcolor=\color{grey},
numbers=left,
numberstyle=\tiny,
numbersep=5pt,
showstringspaces=false,
tabsize=2,
breaklines=true
}


% Title Page
\title{INFO-F403 Introduction to Language Theory and Compilation}
\author{Chapeaux Thomas\\Dagnely Pierre}

\begin{document}
\maketitle


\pagebreak
%%%%%%%%%%%%%%%%%%%%%%%%%%%%%%%%%%%%%%%%%%%%%%%%%%%%%%%%%%%%%%%%%%%%%%%%%%%%%%%%%%%%%%%%%%%%%%%%%%%%%%%%%%%%%%%%%%%%%%%%%%%%%%%%%%%%%%%%%
%%
%%
%%	Quenstion 1
%%
%%
%%%%%%%%%%%%%%%%%%%%%%%%%%%%%%%%%%%%%%%%%%%%%%%%%%%%%%%%%%%%%%%%%%%%%%%%%%%%%%%%%%%%%%%%%%%%%%%%%%%%%%%%%%%%%%%%%%%%%%%%%%%%%%%%%%%%%%%%%%

\hspace{-2.5cm}\begin{tabular}{|c|l|}
\hline
Lexical units  		& regular expressions \\ \hline
INT					& (UNARY+ + UNARY-)(0+1+2+3+4+5+6+7+8+9)* \\ \hline
FLOAT				& (UNARY+ + UNARY-)(0+1+2+3+4+5+6+7+8+9)*.DOT.(0+1+2+3+4+5+6+7+8+9)* \\ \hline
BOOL				& (0+1) \\ \hline
STRING (?)			& '$\Sigma$'  \\ \hline
VALUE				& INT + FLOAT + BOOL + STRING \\ \hline
VARIABLE			& \$STRING \\ \hline
OPERATOR			& ! + * + / + - $+$ + \\ \hline
OPERATOR-COMP		& $<$ + $>$ + $<=$ + $>=$ + == + != + \&\& + $||$ + not + lt + gt + le + ge + eq + ne \\ \hline
UNARY+				& + \\ \hline
UNARY-				& - \\ \hline
EQUAL				& = \\ \hline
DOT					& . \\ \hline
SEMICOLON			& ; \\ \hline
COMA				& , \\ \hline
AND					& \& \\ \hline
OPEN-PAR			& ( \\ \hline
CLOSE-PAR			& ) \\ \hline
OPEN-BRAC			& \{ \\ \hline
CLOSE-BRAC			& \} \\ \hline
DOLLAR				& \$ \\ \hline
OPEN-COND			& IF \\ \hline
CLOSE-COND 			& ELSE \\ \hline
ADD-COND			& ELSE IF \\ \hline
NEG-COND			& UNLESS \\ \hline
RETURN				& return \\ \hline
FUNCT-CREATION		& SUB \\ \hline
FUNCTION-NAME (?)	& string \\ \hline 
FUNCT-CALL			& \&STRING \\ \hline
PERL-FUNCT-NAME		& defined + int + length + scalar + substr + scalar + substr + print\\ \hline
COMM				& \#.STRING \\ \hline


\end{tabular}
~\\

\hspace{-4.5cm}\begin{tabular}{|c|l|}
\hline
EXPRESSION (?)		& VARIABLE OPERATOR VARIABLE   \\
					& EXPRESSION OPERATOR VARIABLE \\ \hline
EXPRESSION-COND (?)	& VARIABLE OPERATOR-COMP VARIABLE   \\
					& EXPRESSION OPERATOR-COMP VARIABLE \\ \hline
ASSIGNATION			& VARIABLE EQUAL VALUE \\ \hline
CONDITION (?)		& ((OPEN-COND+NEG-COND)EXPRESSION-COND OPEN-BRAC INSTRUCTIONS* CLOSE-BRAC\\
					& (ADD-COND EXPRESSION-COND OPEN-BRAC INSTRUCTIONS* CLOSE-BRAC)* \\
					& (CLOSE-COND EXPRESSION-COND OPEN-BRAC INSTRUCTIONS* CLOSE-BRAC))\\
					& + EXPRESSION (OPEN-COND + NEG-COND) EXPRESSION-COND \\ \hline
INSTRUCTIONS		& ((CONDITION SEMICOLON)* + (EXPRESSION SEMICOLON)* + (FUNCTION-CALL \\ 
					& SEMICOLON)* + (ASSIGNATION SEMICOLON)*)* \\ \hline
PARAM				& DOLLAR VARIABLE (COMA DOLLAR VARIABLE)* \\ \hline
USER-FUNCT-CALL		& AND FUNCTION-NAME (OPEN-PAR CLOSE-PAR + OPEN-PAR PARAM CLOSE-PAR  \\
					& + PARAM) SEMICOLON \\ \hline
PERL-FUNCT-CALL		& defined EXPRESSION + int EXPRESSION + length EXPRESSION \\ 
					& scalar EXPRESSION + substr EXPRESSION COMA INT COMA INT \\
					& scalar EXPRESSION + substr EXPRESSION COMA INT  \\
					& + print (?liste de string) \\ \hline
FUNCTION-CALL		& USER-FUNCT-CALL + PERL-FUNCT-CALL \\ \hline
FUNCTION			& FUNCTION-ID FUNCTION-NAME (OPEN-PAR CLOSE PAR + OPEN-PAR PARAM CLOSE-PAR) \\
					& OPEN-BRAC INSTRUCTIONS (RETURN EXPRESSION + RETURN EXPRESSION-COND \\
					& + RETURN VARIABLE) SEMICOLON CLOSE-BRAC \\ \hline
FUNCTION-LIST		& FUNCTION* \\ \hline
PROGRAM				& (FUNCTION-LIST + INSTRUCTIONS)*\\ \hline

					
					
\end{tabular}




(slide 13)
























\end{document}          
